\documentclass[spanish]{beamer}
\usepackage[utf8]{inputenc}
\usepackage[spanish, es-ucroman, es-noquoting]{babel}
\usepackage{amsmath}
\usepackage{textpos} % Provee entorno textblock
\usepackage{mathdots} % Provee macro \iddots
\usepackage{fancyvrb}

%%%%%%%%%%%%%%%%%%%%%%%%%%%%%%%%%%%%%%%%%%%%%%%%%%%%%%%%%%%%%%%%%%%%%%%%%%%%%%%%
% Modo handout (comentar para versión presentación en pantalla/proyector)
% \usepackage{pgfpages}
% \pgfpagesuselayout{4 on 1}[a4paper, landscape, border shrink=5mm]
% \setbeamertemplate{background canvas}{
%     \tikz \draw (current page.north west) rectangle (current page.south east);
% }
%%%%%%%%%%%%%%%%%%%%%%%%%%%%%%%%%%%%%%%%%%%%%%%%%%%%%%%%%%%%%%%%%%%%%%%%%%%%%%%%

% Quitar controles de navegación
\usenavigationsymbolstemplate{}

% Numerar las transparencias
\setbeamertemplate{footline}[frame number]

\title{Programación Funcional}
\subtitle{
  Paradigmas de Lenguajes de Programación \\
  \vspace{2em}
  Prueba de Oposición \\
  Ayudantía de Segunda \\
  Área Teoría
}
\author{Leandro Lovisolo \\ \footnotesize{\texttt{<leandro@leandro.me>}}}
\date{16 de Septiembre de 2015}
\institute{
  Departamento de Computación \\
  Facultad de Ciencias Exactas y Naturales \\
  Universidad de Buenos Aires
}

\begin{document}
\begin{frame}
  \titlepage
\end{frame}

\begin{frame}[fragile]
  \frametitle{Ejercicio}
  \framesubtitle{Práctica 1, ejercicio 16, parte II)}

  Este ejercicio está basado en un esquema particular de recursión que
  representa la técnica algorítmica divide and conquer aplicada a problemas de
  listas. El esquema \texttt{dac}, entonces, tendrá el siguiente tipo e
  implementación:

  \begin{Verbatim}
dac :: b -> (a -> b) -> ([a] -> ([a],[a])) ->
       ([a] -> b -> b -> b) -> [a] -> b
dac base base1 divide combine []    = base
dac base base1 divide combine [x]   = base1 x
dac base base1 divide combine input = combine input
                                      (rec izquierda)
                                      (rec derecha)
  where rec       = dac base base1 divide combine
        izquierda = fst (divide input)
        derecha   = snd (divide input)
  \end{Verbatim}
\end{frame}

\begin{frame}
  \frametitle{Ejercicio}
  \framesubtitle{Práctica 1, ejercicio 16, parte II)}

  Usar \texttt{dac} para implementar el algoritmo de ordenamiento
  \textit{merge-sort}. Dicho algoritmo se comporta de la siguiente manera: se
  divide a la lista a ordernar a la mitad (si es impar, en 2 partes lo más
  iguales posibles), se ordenan esas partes y luego se fusionan (``merge'') en
  orden lineal, tomando iterativamente el más chico de las dos cabezas hasta
  que una lista queda vacía, momento en el cual agrega al final todo lo que
  queda de la otra y devuelve eso como resultado.

  \vspace{1em}

  \texttt{mergeSort :: Ord a => [a] -> [a]}

  \vspace{1em}

  \textbf{Nota}: para hacer la función auxiliar merge se puede usar recursión
  explícita.
\end{frame}

\begin{frame}[fragile]
  \frametitle{Entendiendo el esquema de recursión (I)}

  \begin{Verbatim}
dac :: b                     -- base (lista vacía)
    -> (a -> b)              -- base1 (un sólo item)
    -> ([a] -> ([a],[a]))    -- divide
    -> ([a] -> b -> b -> b)  -- combine
    -> [a]                   -- input list
    -> b                     -- result

dac base base1 divide combine []    = base
dac base base1 divide combine [x]   = base1 x
dac base base1 divide combine input = combine input
                                      (rec izquierda)
                                      (rec derecha)
  where rec       = dac base base1 divide combine
        izquierda = fst (divide input)
        derecha   = snd (divide input)
  \end{Verbatim}
\end{frame}

\begin{frame}[fragile]
  \frametitle{Entendiendo el esquema de recursión (II)}

  Dado que el tipo de nuestra función es el siguiente:

  \vspace{1em}

  \texttt{mergeSort :: Ord a => [a] -> [a]}

  \vspace{1em}

  al aplicarle los parámetros correspondientes, el tipo de la función
  \texttt{dac} se inferiría de la siguiente manera:

  \begin{Verbatim}
dac :: Ord a
    => [a]                         -- base (lista vacía)
    -> (a -> [a])                  -- base1 (un sólo item)
    -> ([a] -> ([a],[a]))          -- divide
    -> ([a] -> [a] -> [a] -> [a])  -- combine
    -> [a]                         -- input list
    -> [a]                         -- result
  \end{Verbatim}
\end{frame}

\begin{frame}[fragile]
  \frametitle{Casos base}

  Recordemos los casos base del esquema de recursión:

  \vspace{1em}

  \begin{Verbatim}
dac :: Ord a
    => [a]         -- base (lista vacía)
    -> (a -> [a])  -- base1 (un sólo item)
    ...
dac base base1 divide combine []    = base
dac base base1 divide combine [x]   = base1 x
...
  \end{Verbatim}

  \vspace{1em}

  Lo que queremos es devolver la lista de entrada tal cual cuando el tamaño es
  menor o igual a 1. Entonces podemos definir los casos base de la siguiente
  manera:

  \vspace{1em}

  \begin{Verbatim}
base    = []
base1 x = [x]
  \end{Verbatim}
\end{frame}

\begin{frame}[fragile]
  \frametitle{Dividir}

  El siguiente paso es proveer la función que dividie el problema original en
  dos problemas más pequeños. Recordemos el tipo de nuestra función
  \texttt{dac}:

  \vspace{1em}

  \begin{Verbatim}
dac :: ...
    -> ([a] -> ([a],[a]))  -- divide
    ...
  \end{Verbatim}

  \vspace{1em}

  Es decir, necesitamos una función que, dada una lista, devuelva dos listas
  más pequeñas. Podemos usar la función \texttt{splitAt} del preludio para
  partir la lista original en dos mitades:

  \vspace{1em}

  \begin{Verbatim}
divide xs = splitAt ((length xs) `div` 2) xs
  \end{Verbatim}
\end{frame}

\begin{frame}
  \frametitle{Conquistar (I)}

  El último paso es la operación \textit{merge} del algoritmo
  \textit{merge-sort}.

  \vspace{1em}

  Dadas dos listas ordenadas, queremos fusionarlas en una
  sola lista ordenada con los elementos de ambas (en tiempo lineal.)
\end{frame}

\begin{frame}[fragile]
  \frametitle{Conquistar (II)}

  Veamos otra vez la definición de \texttt{dac}:

  \vspace{1em}

  \begin{Verbatim}
dac :: ...
    -> ([a] -> [a] -> [a] -> [a])  -- combine
    ...
dac base base1 divide combine input = combine input
                                      (rec izquierda)
                                      (rec derecha)
  where ...
  \end{Verbatim}

  \vspace{1em}

  Notemos que el esquema de recursión requiere un parámetro \texttt{combine}
  que toma la lista original y el resultado de las dos llamadas recursivas, y
  que devuelve el resultado de combinar ambos resultados.

  \vspace{1em}

  En nuestro caso particular no es necesario utilizar la entrada original, por
  lo tanto la ignoramos.
\end{frame}

\begin{frame}[fragile]
  \frametitle{Conquistar (III)}
  Nuestra función \texttt{merge} quedaría definida de la siguiente manera:

  \vspace{1em}

  \begin{Verbatim}
merge _ [] right = right
merge _ left [] = left
merge xs (l:ls) (r:rs)
  | l <= r    = l:(merge xs ls (r:rs))
  | otherwise = r:(merge xs (l:ls) rs)
  \end{Verbatim}
\end{frame}

\begin{frame}[fragile]
  \frametitle{Componiendo la solución}

  \begin{Verbatim}
base = []

base1 x = [x]

divide xs = splitAt ((length xs) `div` 2) xs

merge _ [] right = right
merge _ left [] = left
merge xs (l:ls) (r:rs)
  | l <= r    = l:(merge xs ls (r:rs))
  | otherwise = r:(merge xs (l:ls) rs)

mergeSort :: Ord a => [a] -> [a]
mergeSort = dac base base1 divide merge
  \end{Verbatim}
\end{frame}

\begin{frame}
  \begin{center}
    \Huge{¿Preguntas?}
  \end{center}
\end{frame}
\end{document}

